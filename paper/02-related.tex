\section{Related Work}

The clearest motivation for our platform derives from the current resources available for job seekers in the tech industry. Books such as \cite{McDowell} and \cite{EPI} are seen as the canonical guides to performing well in any interview involving a knowledge of computer science, whether it be a coding interview, brainteasers, or a broader technical discussion with the interviewer. These books are designed as all-in-one guides to scoring a dream job, and so contain an overwhelming amount of material. It can be difficult to pick a starting point, or to identify precisely what topics you should review further. Further, because they are published books, these materials are not as dynamic as an online platform could be (e.g. the question bank is not updated until the next edition). In contrast, this platform can be updated at any point with corrections and new material. Additionally, it will be portable - rather than needing a book, laptop, or tablet physically with you that has the book on it, all that is needed is access to the application. By only seeing one question at a time, the amount of material the user sees at any given time is not overwhelming.

Multiple web and mobile based applications have been developed with the goal of improving the study experience. Quizlet \cite{Quizlet} is a browser-based and mobile application  flashcard tool allowing users to create their own sets of flashcards or search for sets created by other users. Users can have the website generate quiz games with the flashcard set (such as fill-in-the-blank or matching) to make the experience more interactive. During these games, Quizlet keeps track of cards the user gets wrong, building the next round of the game entirely from the cards they answered incorrectly during the current round. When a user is flipping through the set as they would with physical cards, they have the option to flag cards they are having trouble with. Ideally, our platform to have a similar function in which users can flag questions they would like to come back to. This saved application state would require a database look up for the flagged questions every time the user returns.

The potential player-vs.-player feature that we hope to implement is inspired by QuizUp \cite{Quizup}. Quizup a popular mobile trivia application that allows users to challenge each other, earn badges, and choose from over 400 categories. We would like to draw from the competitive nature of this app for our own application, as we believe that adding a social aspect will increase collaboration and motivation when studying.